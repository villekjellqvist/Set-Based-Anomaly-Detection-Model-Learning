% !TEX root = root.tex

% Figures
\usepackage{tikz}

% Maths
\usepackage{amsmath, amsfonts, amssymb}
\usepackage{xargs}
% Algorithms
\usepackage[linesnumbered,ruled]{algorithm2e}
\usepackage{algpseudocode}
\SetKwInOut{Input}{Input}
\SetKwInOut{Output}{Output}

% Comments
\usepackage[colorinlistoftodos]{todonotes}
\newcommand{\AT}[1]{\todo[inline, color=orange!60]{\textbf{AT:} #1}}
\newcommand{\VK}[1]{\todo[inline, color=gray!60]{\textbf{VK:} #1}}

%Citations
\usepackage[url=false, style=ieee]{biblatex}
\addbibresource{MHE-Hyptest.bib}

% Assumptions
\newtheorem{assumption}{Assumption}
\newtheorem{lemma}{Lemma}
\newtheorem{remark}{Remark}

% Symbol Defines
\newcommand{\Mp}{\mathcal{M}_{\text{prior}}}
\newcommand{\nullA}{\mathcal{K}_A}
\newcommand{\Pran}{\Pi_A}
\newcommand{\Port}{\Pi_A^\perp}
\newcommand{\tC}{\hat{\theta}_C}
\newcommand{\Pinv}{P_0^{-1}}
\newcommand{\PC}{\overline{P}_0^{-1}}
\newcommand{\QC}{\nullA^TQ\nullA}

\newcommand{\hyp}{\mathcal{H}}
\newcommand{\tU}{\theta_{U}}
\newcommand{\LH}{\mathcal{L}}

% For wilks proof
\newcommand{\hess}{\mathcal{H}}
\newcommand{\omeganu}[5][\\]{\begin{bmatrix}
    #2{\omega}#4 #1 #3{\nu}#5 
\end{bmatrix}}

\newcommandx{\omeganudiff}[2][1=\omega, 2=\nu]{\begin{bmatrix}
    \omega-\hat{\omega} \\ \tilde{\nu}-\hat{\nu}
\end{bmatrix}}
\newcommand{\partialhess}{\begin{bmatrix}
            I_{n_e} \\ -\hess_{\nu\nu}^{-1}\hess_{\nu\omega}
        \end{bmatrix}}